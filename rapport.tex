\documentclass[a4paper,12pt]{article}
\usepackage[utf8]{inputenc}
\usepackage[T1]{fontenc}
\usepackage{setspace}
\usepackage[english]{babel} % Définit la langue en anglais
\usepackage{lmodern}
\usepackage{microtype}
\usepackage{amsmath,amsthm}
\usepackage{amsfonts}
\usepackage{amssymb}
\usepackage{graphicx}
\usepackage{hyperref}
\usepackage{textpos}
\usepackage{dirtytalk}
\usepackage{fancyhdr}
\usepackage[nottoc]{tocbibind}
\usepackage{float}
\usepackage{enumitem}
\usepackage{tikz}
\usepackage{geometry}
\usepackage{array}
\usepackage[pages=some]{background}
\usepackage{mathrsfs}
\usepackage{algorithm}
\usepackage{algorithmic}
\usepackage{longtable}
\usepackage{tikz}
\usetikzlibrary{calc}
\usepackage{multirow}
\usepackage{makecell}
\usepackage{amsmath}
\usepackage{indentfirst}
\usepackage{caption}
%\captionsetup[table]{position=bottom}
\usepackage{siunitx}
\usepackage{titlesec}
\numberwithin{equation}{section}
\usepackage[utf8]{inputenc}
\usepackage{amsmath}
\usepackage{listings}
\usepackage{xcolor}
\usepackage{graphicx}
\usepackage{ulem}
\usepackage{amssymb}
\usepackage{epsfig}
\usepackage{epstopdf}
\usepackage{hyperref}  % Pour liens cliquables
\usepackage{tocloft}      % Pour personnaliser la table des matières
\usepackage[nameinlink,noabbrev]{cleveref}
\usepackage{url}
\usepackage{xcolor}
\usepackage{booktabs}
\usepackage{biblatex}
\addbibresource{references.bib}




\lstset{
  language=Python,
  basicstyle=\ttfamily\footnotesize,
  keywordstyle=\color{blue},
  commentstyle=\color{gray},
  stringstyle=\color{red},
  showstringspaces=false,
  tabsize=4,
  breaklines=true,
  caption=RTC Initialization (Set Time),
  frame=single
}

\sisetup{per-mode=symbol,detect-all}
\setlist[itemize]{topsep=2pt,itemsep=2pt,parsep=0pt}
\setlist[enumerate]{topsep=2pt,itemsep=2pt,parsep=0pt}

% ====== Helpful macros ======
\newcommand{\vlogic}{\SI{3.3}{V}}
\newcommand{\vfield}{\SI{24}{V}}
\newcommand{\imaxout}{\SI{500}{mA}}
\lstset{
  language=Python,
  basicstyle=\ttfamily\footnotesize,
  keywordstyle=\color{blue},
  commentstyle=\color{gray},
  stringstyle=\color{red},
  showstringspaces=false,
  tabsize=4,
  breaklines=true,
  caption=RTC Initialization (Set Time),
  frame=single
}

% Configuration pour la numérotation des pages en bas
\fancyhf{} % Efface les en-têtes et pieds de page par défaut
\fancyfoot[C]{\thepage} % Place le numéro de page au centre du pied de page
\pagestyle{fancy} % Applique le style fancy pour tout le document
% Pour changer le titre de la table des matières
\addto\captionsenglish{%
  \renewcommand{\contentsname}{Table of Contents}%
  \renewcommand{\listfigurename}{List of Figures}%
}


%-------------------------------

\clearpage

\newpage
%----------------------------------------
\setcounter{secnumdepth}{3}

\setcounter{tocdepth}{3}


\renewcommand{\cfttoctitlefont}{\hfill\bfseries}
\renewcommand{\cftaftertoctitle}{\hfill}

\begin{document}


\tableofcontents
\newpage

\section{Introduction}
Le projet \textbf{Thor-ROS} consiste en la simulation et le contrôle du bras robotique \textbf{Thor Arm} sous ROS2 et Gazebo.  
L'objectif principal est de réaliser des opérations de type \textit{pick and place} et de tester le robot dans un environnement simulé avant de le contrôler sur le robot réel.  

---

\section{Structure générale du projet}

Le projet est organisé en plusieurs packages ROS2. Le workspace se trouve à :

\begin{lstlisting}[language=bash]
~/thor_arm_ws/src/Thor-ROS/ws_thor/src/
\end{lstlisting}

Les principaux packages sont :

\begin{itemize}
    \item \textbf{thor\_bringup} : contient les fichiers de lancement pour la simulation et la communication avec le robot.
    \item \textbf{thor\_moveit} : configuration MoveIt pour le contrôle du bras, planification de trajectoire et GUI RViz.
    \item \textbf{thor\_urdf} : description du robot en URDF, incluant tous les liens et joints.
    \item \textbf{thor\_arm\_description} : modèle visuel et collision du bras (textures, meshes).
    \item \textbf{thor\_controller} : configuration des contrôleurs de mouvement ROS2 pour les joints.
    \item \textbf{thor\_status\_publisher} : publie l'état du robot (position, joints, capteurs) sur ROS2.
    \item \textbf{thor\_server} : serveur ROS2 pour recevoir et exécuter les commandes.
    \item \textbf{thor\_msgs} : messages ROS2 personnalisés pour la communication interne.
    \item \textbf{thor\_joystick} : interface pour contrôler le bras via un joystick.
\end{itemize}


\section{Description des fichiers importants}

\subsection{thor\_bringup/launch}
\begin{itemize}
    \item \textbf{real\_robot.launch.py} : lancement du robot physique réel.
    \item \textbf{simulated\_robot.launch.py} : lancement de la simulation Gazebo avec tous les nœuds nécessaires.
    \item \textbf{rosbridge.launch.py} : lance le serveur ROSBridge pour la communication avec des interfaces web.
\end{itemize}

\subsection{thor\_urdf}
\begin{itemize}
    \item Contient le fichier \textbf{thor.urdf} décrivant la structure mécanique du bras (liens et articulations).
    \item Définit les limites des joints, les inerties et les collisions.
\end{itemize}

\subsection{thor\_moveit/config}
\begin{itemize}
    \item \textbf{thor.srdf} : configuration SRDF pour MoveIt, définissant les groupes de joints et les poses prédéfinies.
    \item Configurations des planificateurs de trajectoire, contrôleurs et paramètres MoveIt.
\end{itemize}

\subsection{thor\_controller/config}
\begin{itemize}
    \item Définit les contrôleurs ROS2 pour les joints : position, vitesse et effort.
\end{itemize}

\subsection{thor\_msgs/msg}
\begin{itemize}
    \item Contient des messages ROS2 personnalisés pour la communication interne.
\end{itemize}

\subsection{thor\_status\_publisher}
\begin{itemize}
    \item Publie périodiquement les positions et états du robot sur les topics ROS2.
\end{itemize}

\subsection{thor\_joystick}
\begin{itemize}
    \item Permet le contrôle manuel du bras via un joystick.
\end{itemize}

---

\section{Comment comprendre et utiliser le projet}

\begin{enumerate}
    \item \textbf{Simulation Gazebo} : lancer \texttt{simulated\_robot.launch.py} depuis \texttt{thor\_bringup/launch}.
    \item \textbf{Contrôle du bras} : utiliser MoveIt avec RViz en lançant \texttt{thor\_moveit.launch.py}.
    \item \textbf{Contrôleurs} : configurés dans \texttt{thor\_controller/config} pour gérer le mouvement réel ou simulé.
    \item \textbf{Messages et états} : consultables via les topics ROS2 publiés par \texttt{thor\_status\_publisher} et \texttt{thor\_msgs}.
    \item \textbf{Joystick} : interface manuelle de contrôle si nécessaire.
\end{enumerate}

---

\section{Résumé}
Le projet Thor-ROS est structuré de manière modulaire pour séparer :
\begin{itemize}
    \item La simulation et le lancement (\texttt{thor\_bringup})
    \item La description mécanique (\texttt{thor\_urdf}, \texttt{thor\_arm\_description})
    \item Le contrôle et planification (\texttt{thor\_controller}, \texttt{thor\_moveit})
    \item Les messages et états (\texttt{thor\_msgs}, \texttt{thor\_status\_publisher})
    \item Les interfaces manuelles (\texttt{thor\_joystick}, \texttt{rosbridge\_server})
\end{itemize}

Cette structure permet de tester et contrôler le Thor Arm dans la simulation avant de le connecter au robot réel.

\end{document}
